% !TEX encoding = UTF-8 Unicode
% !TEX spellcheck = en-US

\documentclass[10pt,a4paper,titlepage]{article}
\usepackage[utf8]{inputenc}
\usepackage[T1]{fontenc}
\usepackage{amsmath}
\usepackage{ragged2e}
\usepackage{hyperref}
\usepackage{amsfonts}
\usepackage{amssymb}
\usepackage{graphicx}
\usepackage{xcolor}
\usepackage{listings}
\usepackage{minted}
\hypersetup{
	colorlinks=true,
	linkcolor=blue,
	filecolor=magenta,      
	urlcolor=cyan,
}
\definecolor{codegreen}{rgb}{0,0.6,0}
\definecolor{codegray}{rgb}{0.5,0.5,0.5}
\definecolor{codepurple}{rgb}{0.58,0,0.82}
\definecolor{backcolour}{rgb}{0.95,0.95,0.92}

\lstdefinestyle{mystyle}{
	backgroundcolor=\color{backcolour},   
	commentstyle=\color{codegreen},
	keywordstyle=\color{magenta},
	numberstyle=\tiny\color{codegray},
	stringstyle=\color{codepurple},
	basicstyle=\ttfamily\footnotesize,
	breakatwhitespace=false,         
	breaklines=true,                 
	captionpos=b,                    
	keepspaces=true,                 
	numbers=left,                    
	numbersep=5pt,                  
	showspaces=false,                
	showstringspaces=false,
	showtabs=false,                  
	tabsize=2
}

\lstset{style=mystyle}

\urlstyle{same}
\author{Yapi D}
\title{Lab 1}


\begin{document}
	\makeatletter
\begin{titlepage}
	\begin{center}
		\includegraphics[width=0.7\linewidth]{pylogo}\\[4ex]
		{\huge \bfseries  \@title }\\[2ex] 
		{\LARGE  \@author}\\[50ex] 
		{\large \@date}
	\end{center}
\end{titlepage}
\makeatother
\thispagestyle{empty}
\newpage

\section{Objective}
In this first lab you will install all tools needed to program in python and learn
about python \textcolor{blue}{function}, \textcolor{blue}{function arguments}, \textcolor{blue}{variable} and learn about two pythons packages:
\textcolor{blue}{Matplotlib} for plotting and \textcolor{blue}{Pandas} for reading csv/excel files. I will explain the concept of python package as well.
	\section{Installing all requirements}
	The first step is to install all the requirements we need to program in Python. You start by installing \textcolor{blue}{Python}, the text editor \textcolor{blue}{Atom}. Then you create a \textcolor{blue}{Github account}. Github is a cloud server that allows you to save your code safely in a cloud environment. This is how modern software development is done. It allows you to cooperate with other software developers, by safely saving and sharing your code.
	
	\justify
	Follow these instructions:
	\justify
	\begin{enumerate}
		\item \href{https://www.python.org/downloads/release/python-374/}{Click here to install Python}: Downloads > Download Python 3.7.4. Make sure to select \textcolor{blue}{Add Python path}, and follow the instructions.
		\item Install the text editor atom \href{https://atom.io/}{ from here}
		\item Get a github account \href{https://github.com/}{from here}
		\item Now go into your github account and create a repository called yourname$\_$python$\_$course
		\item Install git on window \href{https://git-scm.com/downloads}{from here}
		\item Invite me as a collaborator by clicking : Settings > Collaborators,  then add my github username \textcolor{blue}{yellowsimulator} in the box and click \textcolor{blue}{Add collaborator}
	\end{enumerate}
\subsection{Important linux and Window Powershell commands}
\begin{lstlisting}[language=bash]
#create a folder called myfolder
$ mkdir myfolder
\end{lstlisting}

\begin{lstlisting}[language=bash]
#create a a file called myfile.py
$ New-Item myfile.py
\end{lstlisting}

\begin{lstlisting}[language=bash]
#delate the file called myfile.py
$ rm myfile.py
\end{lstlisting}


\begin{lstlisting}[language=bash]
#go inside the folder called myfolder
$ cd myfolder
\end{lstlisting}

\begin{lstlisting}[language=bash]
#see what is inside the folder called myfolder
$ ls myfolder
\end{lstlisting}

\begin{lstlisting}[language=bash]
#go back
$ cd .. 
\end{lstlisting}

\subsection{Importants git commands (git status, git add, git commit, git push)}
\begin{lstlisting}[language=bash]
#create a local copy of a folder call somerepository
$ git clone somerepository
\end{lstlisting}
\justify
To save your code on git
\begin{lstlisting}[language=bash]
#check all files that chaged
$ git status

#add file called filename
$ git add filename

#confirm that you added the file 
$ git commit -m "add your message here"

#now save your file on the cloud
$ git push 
\end{lstlisting}


\section{Installing more python stuff}
Now we will create a \textcolor{blue}{virtual environment}. A virtual environment allows you to manages all your python code nicely. You will learn why it is important.
\begin{lstlisting}[language=python]
#create a virtual environment called km_python_env
$ python -m venv km_python_env
\end{lstlisting}

\begin{lstlisting}[language=bash]
#activate the virtual environement
$ km_python_env\Scripts\activate
\end{lstlisting}

\begin{lstlisting}[language=bash]
#install numpy, matplotlib, pandas
$ pip install numpy
$ pip install matplotlib
$ pip install pandas
\end{lstlisting}


\subsection{Exercise}
\begin{enumerate}
	\item create a local copy of the github repository you created before on you computer (use git clone)
	\item go inside that repository and create a folder called lab1 (use cd and mkdir)
	\item now create the file python$\_$lab1.py inside the folder lab1 (use New-Item)
\end{enumerate}

\section{Function, variable, arguments and reading files}
\subsection{Function, variable and arguments}
\begin{lstlisting}[language=python]
#create a function in python 
def my_function():
	"""
	This is my cool function
	"""
	print("I am awesome")
\end{lstlisting}

\justify
Functions allows you to group your code that executes one task at the time. A variable holds a value of something.
The message at the beginning of the function between """   """ is called \textcolor{blue}{doctrings}. Why do we need it?
\clearpage
\begin{lstlisting}[language=python]
#create a function with a variable holding a string 
def print_message():
	"""
	This is my cool function
	with a variable
	"""
	my_variable = "I am awesome"
	print(my_variable)
\end{lstlisting}
\justify

\begin{lstlisting}[language=python]
#create a function with one argument 
def print_message(my_variable):
	"""
	This is my cool function
	with one argument
	"""
	print(my_variable)

#Initialize variable and call the function
my_variable = "I am awesome"
print_message(my_variable)
\end{lstlisting}
\justify
\begin{lstlisting}[language=python]
#create a function with two arguments
def print_message(name, job):
	"""
	This is my cool function
	with two arguments.
	"""
	message = "My name is {}, I am a {}".format(name,job)
	print(message)

#Initialize variables and call the function
name = "Yapi Thor "
job = " Musician"
print_message(name, job)
\end{lstlisting}

\begin{lstlisting}[language=python]
#create a function with many arguments
def print_arguments(**args):
"""
This function takes as many arguments
as you want 
"""
print(args)

#Initialize variables and call the function
name1 = "bla"
name2 = "blabla"
print_arguments(name1, name2)
\end{lstlisting}


\subsection{Importing python packages and reading a csv/exl file}
Here you are introduced to the python package called \textcolor{blue}{pandas} for reading data from a csv file or excel file and \textcolor{blue}{matplotlib} for plotting the data. To read data from a csv or excel file use the function \textcolor{blue}{read$\_$csv()} and \textcolor{blue}{read$\_$excel()}
\begin{lstlisting}[language=python]
#A function that plots a freqeuncy spectrum
import pandas as pd
import matplotlib as plt


def plot_frequency_spectrum():
	"""
	plot a frequency spectrum
	"""
	path = "data.csv"
	df = pd.read_csv(path)
	frequency = df["frequency"].values
	amplitude = df["amplitude"].values
	plt.plot(frequency, amplitude)
	plt.xlabel("Frequency")
	plt.ylabel("Amplitude")
	plt.title("Frequency spectrum")
	plt.show()
	
# plot_frequency_spectrum()
\end{lstlisting}

\subsubsection{Exercise}
This exercise will introduce the \textcolor{blue}{if, else} statement
\begin{enumerate}
	\item Update the above function to take the \textcolor{blue}{path}, \textcolor{blue}{x label}, \textcolor{blue}{y label} and \textcolor{blue}{title} as arguments.
	\item Now write a function that can plot data either from a csv (\textcolor{blue}{data.csv}) file or an excel file \textcolor{blue}{data.xls}.
	\begin{lstlisting}[language=python]
	#some hint
	def my_function(path, ...):
		"""
		This is my cool function
		"""
		if  path == "data.csv":
			df = pd.read_csv(path)
		else:
			df = pd.read_excel(path)
		
	\end{lstlisting}
\end{enumerate}

\end{document}